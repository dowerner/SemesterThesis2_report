\chapter{Summary and Outlook}
The goal of this work was to develop and evaluate a method to fabricate spherical microcavity mirrors. The purpose of theses mirrors is their utilization as part of a microcavity for the use in a trapping experiment. The small mode volume of the microcavity would allow a high sensitivity enabling the position measurement of a levitated nanosphere.

The process was developed and the various steps have been tested and evaluated. The melting of standard communication glass fibers into spherical glass stamps was achieved with the construction of a rotating fixture and the utilization of a small blow torch for melting. The resulting stamps were measured to determine the ROC which has to be between $0.7\mm$ and $1.0\mm$. All produced stamps had the proper dimensions. Furthermore, the stamps were measured with the AFM to determine their surface roughness since the RMS value is a key quantity in assessing losses of the final mirrors. AFM measurements showed that the glass stamps as well as the OrmoComp polymer used to imprint the mirrors into have $\si{RMS}<0.4\nm$ which meets the requirement which was set to be an RMS smaller than $0.6\nm$.

To preserve the good surface quality of both stamp and polymer during imprinting a silanization treatment was evaluated and tested to add an anti-adhesive layer to the stamp. Tests without the application of this layer resulted in the stamps ripping apart when trying to remove them from the cured mirrors.

When using the stamps to imprint the mirrors into the polymer a problem was discovered that proved to be difficult to be localized. The mirrors produced were always too wide and too deep, which rendered them unusable for the trapping experiment. Tests with different imprinting methods finally led to the conclusion that the problem was in fact the polymer creeping up the glass stamps during imprinting. This behavior was also confirmed by the manufacturer of the polymer. 

Grinding the mirrors with sandpaper has been shown to be a possible solution to get the mirrors to the proper dimensions. However, grinding was deemed to be very hard to control and greatly reduced the yield of produced mirrors. Of the four successfully produced mirrors all had $\si{RMS}\approx 1.7\nm$ which is about three times higher than required. Identifying the grinding procedure as the likely cause of this problem led to the conclusion that omitting this step altogether would be an important obstacle to overcome.

To overcome this obstacle a new method for imprinting was developed which uses additional glass structures to contain the polymer in the volume required. The fabrication of this new stamp was partly tested but due to time constraints the real evaluation will be the goal of a future project.