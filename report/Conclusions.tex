\chapter{Summary and Outlook}
The goal of this work was to design and build a narrow linewidth $\kappa < 2\pi\times 10\kHz$ optical cavity able to resolve the motional sidebands of an optically levitated nanosphere. Furthermore, the cavity resonance needed be stable enough to allow to lock a laser to the cavity in a stable manner.

During the design process, particular attention was put on obtaining a cavity with a stable resonance frequency, by minimizing the influence of environmental disturbances (in particular thermal and mechanical noise) to the cavity resonance. To this end, in the cavity construction, materials with low thermal expansion rates were employed and an active stabilization of the cavity temperature was implemented. Furthermore, a vibration isolation system as well as a vacuum setup were built.

A laser was successfully frequency locked to the cavity using the Pound-Drever-Hall locking technique. The laser was stabilized to the cavity resonance more precisely than $\kappa$ and the lock remained stable against perturbations (knocking on the optic table, clapping), proving that the isolation of the cavity from environmental mechanical and acoustic noise was efficient enough. To this point it was not possible to quantitatively characterize the stability of the cavity resonance with respect to temperature fluctuations in the environment, i.e. determine the coefficient of thermal expansion of the mirror spacer. As soon as a stable (absolute) frequency source is available this value will become accessible. Being able to stabilize the laser to the cavity was a first step towards a frequency locked system consisting of microcavity, external cavity and optical field (Fig. \ref{fig:sidebands}).

Finally, important parameters characterizing the cavity were determined. Among them, the cavity linewidth was found to be $\kappa \approx 2\pi\times40.34(4) \kHz$, a larger value than the linewidth that was computed based on the mirror specifications. Consistently, the absorption in the cavity mirrors was measured to be higher ($\mathcal{A} \approx 122(6)$ ppm) than expected ($\mathcal{A}<1$ ppm). Determining the reasons for the increased absorption $\mathcal{A}$ remains a task for the future.
