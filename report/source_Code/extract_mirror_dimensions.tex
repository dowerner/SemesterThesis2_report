\begin{lstlisting}
"""
File:           extract_mirror_dimensions.py
Author:         Dominik Werner, 2018
Description:    This script opens calculates the mirror dimensions from measured radii of fabricated stamps
"""

import numpy as np

# input and output files
input_file_path = 'Samples/05.12.2018/Results/radii.csv'
result_file_path = 'Samples/05.12.2018/Results/calculated_mirrors.csv'

# Definitions
desired_cavity_length = 500e-6  # 0.5 mm
cover_slip_thickness = 160e-6  # https://de.vwr.com/store/product/3006241/deckglaeser-menzel-glaeser
ormocomp_layer_thickness = 25e-6  # rough estimate
diameter_factor = 5

pi = np.pi
wl = 1565e-9    # wavelength

focus_waist_from_r_L = lambda R, L: np.sqrt(wl/pi)*(L*(R-L))**(1/4)


def extract_mirror(ROC, name):
    '''
    Extract mirror dimensions from a given ROC
    :param ROC: radius of curvature in m
    :param name: name of the sample
    :return: beam waist at the focus point, beam waist radius at mirror position,
    Gaussian beam diameter on mirror, mirror diameter, mirror depth
    '''
    w0 = focus_waist_from_r_L(ROC, desired_cavity_length)  # beam waist at focus
    k = 2*pi/wl     # wave vector
    zR = pi*w0**2/wl    # Rayleigh range
    n = 1   # refractive index
    th = wl/(pi*n*w0)   # divergence angle
    psi = lambda z: np.arctan(np.divide(z, zR))    # Gouy phase
    R = lambda z: z*(1 + np.power(np.divide(zR, z), 2))     # wave front radius
    z_from_R = lambda R: (R+np.sqrt(R**2-4*zR**2))/2
    w = lambda z: w0*np.sqrt(1 + np.power(np.divide(z, zR), 2))     # beam width

    r1 = ROC  # 1mm
    z1 = z_from_R(r1)
    D = w(z1 + ormocomp_layer_thickness + cover_slip_thickness) * diameter_factor
    D0 = 2*r1*np.arctan(w(desired_cavity_length)/r1)  # Mirror diameter hit by gaussian beam
    h1 = r1 - np.sqrt(r1**2 - D**2/4)  # height of desired mirror

    print('Mirror:', name)
    print('Wavefront Radius: R1 =', r1*1e6, 'µm')
    print('Desired hemispherical cavity length: L =', desired_cavity_length*1e6, 'µm')
    print('Required focal radius: w0 =', w0*1e6, 'µm')
    print('Mirror-Center Distance: d1 =', z1*1e6, 'µm')
    print('Mirror diameter hit by gaussian beam: D0 =', D0*1e6, 'µm')
    print('Desired mirror diameter (opening circle): D =', D*1e6, 'µm')
    print('Height/Depth of desired mirror: h1 =', h1*1e6, 'µm')
    print('Focus width (diameter): w0 =', 2*w0*1e6, 'µm')
    print('Cavity length (hemispherical): L = ', z1*1e6, 'µm')
    print('***************************************************************************************')

    return w0, z1, D0, D, h1


input_file = open(input_file_path, 'r')
result_file = open(result_file_path, 'w+')

result_file.write('Sample;Radius [µm];Cavity Length [µm];Required w0 [µm];'
                  'Beam Diameter on mirror [µm];Mirror Diameter [µm];Mirror Depth [µm]\n')

lines = input_file.readlines()  # read lines with the radii of the measured stamps
for i in range(1, len(lines)):   # skip first line
    line = lines[i]
    cells = line.split(';')
    if len(cells) == 2:
        name = cells[0]
        R = float(cells[1])*1e-6
        w0, L, D0, D, h1 = extract_mirror(R, name)
        result_file.write('{};{};{};{};{};{};{}\n'.format(name, R*1e6, L*1e6, w0*1e6, D0*1e6, D*1e6, h1*1e6))

result_file.close()
input_file.close()
\end{lstlisting}

