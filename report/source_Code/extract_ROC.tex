\UseRawInputEncoding
\begin{lstlisting}
"""
File:           extract_ROC.py
Author:         Dominik Werner, 2018
Description:    This script opens images taken on the microscope of the fabricated stamps and tries to extract the ROC
"""

import numpy as np
import matplotlib.pyplot as plt
from PIL import Image
from skimage import measure
from scipy.optimize import curve_fit
from skimage import feature, color
import os

# import settings
date = '05.12.2018'
sample = 'tip4'

show_single_image = False   # if set to true only opens one image and displays the data (else process entire folder)
summary_file = 'radii.csv'  # file to save ROC data to for further processing

required_surface_diameter = 0.3  # mm, used for identifying ROC extraction contour

scale = 520.0010    # px/mm, metric


def find_sample(image):
    '''
    Locate the sample with the molten glas droplet in the image.
    :param image: image
    :return: contour that is most likely the droplet as well as the center of the contour and its width and height.
    '''
    contours = measure.find_contours(image[..., 0], 115)    # search contours
    radius = 0
    for c in contours:  # loop through all the contours to find the largest -> most likely to be the droplet
        dev_h, dev_w = np.std(c, axis=0)    # standard deviation determines the extent of the droplet
        r = np.sqrt(dev_w**2 + dev_h**2)    # calculate the radius of the found contour
        if r > radius:  # if the radius is bigger than the largest up to this point select this candidate
            radius = r
            contour = c
            width = dev_w * 2
            height = dev_h * 2
    y0, x0 = np.mean(contour, axis=0)
    return contour, x0, y0, width, height


def find_circle(image):
    '''
    Use canny edge detection and the ransac algorithm to detect a circle in a given image.
    :param image: image containing the circle
    :return: center point coordinates x and y as well as the radius
    '''
    edges = feature.canny(color.rgb2gray(image), sigma=2)
    '''plt.figure()
    plt.imshow(edges)
    plt.show()'''
    points = np.array(np.nonzero(edges)).T
    model_robust, _ = measure.ransac(points, measure.CircleModel, min_samples=3, residual_threshold=2, max_trials=1000)
    cy, cx, r = model_robust.params
    return cx, cy, r


def circle(x0, y0, r):
    '''
    Outputs a collections of x and y coordinates for plotting a circle.
    :param x0: center point x coordinate
    :param y0: center point y coordinate
    :param r: radius
    :return: x, y collections
    '''
    phi = np.linspace(0, 2*np.pi)
    x = x0 + r * np.cos(phi)
    y = y0 + r * np.sin(phi)
    return x, y


def do_sphere_analysis(path):
    print('opening image...')
    im = np.array(Image.open(path))     # load selected image

    # find sample in the image (the largest contour that can be found)
    print('searching for sample in image...')
    contour, x0, y0, width, height = find_sample(im)

    # define region of interest
    expansion_factor = 1.5

    x_min = x0 - width / 2 * expansion_factor
    x_max = x0 + width / 2 * expansion_factor
    y_min = y0 - height / 2 * expansion_factor
    y_max = y0 + height / 2 * expansion_factor

    # make sure that the indices for the region of interest are bounded
    print('determining region of interest...')
    image_height, image_width, _ = im.shape
    if x_min < 0: x_min = 0
    if y_min < 0: y_min = 0
    if x_max > image_width - 1: x_max = image_width - 1
    if y_max > image_height - 1: y_max = image_height - 1

    # extract region of interest from image for circle detection
    image_roi = im[int(y_min):int(y_max), int(x_min):int(x_max)]

    # extract the contour that lies within the region of interest
    print('extracting contour...')
    mask = (contour[:, 1] > x_min) & (contour[:, 1] < x_max) & (contour[:, 0] > y_min) & (contour[:, 0] < y_max)
    contour_x = contour[mask, 1]
    contour_y = contour[mask, 0]

    # find the angle in which the sample is oriented
    print('estimating droplet center...')
    roi_intersection1 = np.asarray([contour_x[0], contour_y[0]])
    roi_intersection2 = np.asarray([contour_x[-1], contour_y[-1]])
    diff_vec = roi_intersection2 - roi_intersection1
    sample_orientation = np.arctan2(diff_vec[1], diff_vec[0]) + np.pi / 2   # +90° because the orientation is perpendicular
    print('The sample seems to be oriented at an angle of {}°.'.format(round(sample_orientation * 180 / np.pi), 2))

    # rotate (around roi center) the contour such that the maximum of the x part can be used to estimate the center of the droplet
    contour_x_rotated = np.cos(-sample_orientation) * (contour_x - x0) - np.sin(-sample_orientation) * (contour_y - y0) + x0
    contour_y_rotated = np.sin(-sample_orientation) * (contour_x - x0) + np.cos(-sample_orientation) * (contour_y - y0) + y0

    # find cap of the droplet for the circle fit -> around 300µm of surface diameter are needed in the end
    droplet_center_index = np.argmax(contour_x_rotated)     # get contour index of estimated droplet center
    droplet_center_x = contour_x[droplet_center_index]      # get droplet center x coordinate
    droplet_center_y = contour_y[droplet_center_index]      # get droplet center y coordinate

    safety_factor = 2     # expand the fitting area or shrink it if necessary
    diameter = required_surface_diameter * safety_factor    # required surface diameter of the droplet
    threshold_pixel_count = 4   # additional pixels around the circle fit region of interest to make sure nothing is cut off

    contour_start_index = int(droplet_center_index - diameter / 2 * scale)  # fit contour start index
    contour_end_index = int(droplet_center_index + diameter / 2 * scale)    # fit contour end index
    fit_contour_x = contour_x[contour_start_index:contour_end_index]    # x coordinates of the fit contour
    fit_contour_y = contour_y[contour_start_index:contour_end_index]    # y coordinates of the fit contour
    fit_roi_x_min = np.min(fit_contour_x) - threshold_pixel_count
    fit_roi_x_max = np.max(fit_contour_x) + threshold_pixel_count
    fit_roi_y_min = np.min(fit_contour_y) - threshold_pixel_count
    fit_roi_y_max = np.max(fit_contour_y) + threshold_pixel_count
    roi_fit = im[int(fit_roi_y_min):int(fit_roi_y_max), int(fit_roi_x_min):int(fit_roi_x_max)]  # roi for the circle fit

    #plt.figure()
    #plt.imshow(roi_fit)

    # run circle detection on region of interest (is faster and more accurate than on original image)
    print('fitting circle...')
    cx, cy, radius = find_circle(roi_fit)
    cx += int(fit_roi_x_min)    # translate x coordinate of center point to original image
    cy += int(fit_roi_y_min)    # translate x coordinate of center point to original image
    circle_x, circle_y = circle(cx, cy, radius)   # get coordinates of circle for plotting


    fig = plt.figure()    # create figure
    plt.tight_layout()
    plt.title('radius={} mm'.format(round(radius / scale, 2)))
    ax1 = plt.subplot(1, 1, 1)
    ax1.set_xlim([0, image_width / scale])  # set scaling for x axis -> use mm
    ax1.set_ylim([0, image_height / scale])   # set scaling for y axis -> use mm
    ax1.set_xlabel('x [mm]')
    ax1.set_ylabel('y [mm]')
    ax2 = ax1.twinx().twiny()  # instantiate a second axes
    ax2.imshow(im)  # display image

    # draw the region of interest
    ax2.add_patch(plt.Rectangle((x_min, y_min), width*expansion_factor, height*expansion_factor, fill=False, edgecolor='g'))

    ax2.plot(contour_x, contour_y, color='r')   # draw detected contour
    #ax2.plot(contour_x_rotated, contour_y_rotated, color='b')   # draw detected contour (rotated)
    ax2.plot(circle_x, circle_y)    # draw detected circle
    ax2.plot([droplet_center_x], [droplet_center_y], marker='o')    # draw estimated center of droplet
    ax2.plot(fit_contour_x, fit_contour_y, color='y')

    return fig, radius / scale * 1e3     # return radius in µm


if show_single_image:
    do_sphere_analysis('Samples/{}/{}.TIF'.format(date, sample))
    plt.show()
else:
    csv = open('Samples/{}/Results/{}'.format(date, summary_file), 'w+')
    csv.write('Sample;Radius [µm]\n')

    for file in os.listdir('Samples/{}'.format(date)):
        if not os.path.isdir('Samples/{}/{}'.format(date, file)):
            print('Analyzing image: {} ...'.format(file))
            fig, r_um = do_sphere_analysis('Samples/{}/{}'.format(date, file))
            fig.savefig('Samples/{}/Results/{}.png'.format(date, file))
            csv.write('{};{}\n'.format(file.lower().replace('.tif', '').replace('.png', ''), r_um))

    csv.close()
\end{lstlisting}