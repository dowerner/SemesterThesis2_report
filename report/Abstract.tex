\section*{Abstract}
For exploring the boundaries between the micro- and the macroscopic world we work with mesoscopic glass particles. A nano-scale glass particle can be detached from most environmental influences by putting it into a vacuum and using a laser beam to levitate it. This makes it a highly sensitive force sensor, capable of measuring forces as small as $2\cdot 10^{-20}\,\si{N/\sqrt{Hz}}$ \cite{gieseler2013thermal}. The suspension of the particle does not rely on radiation pressure but on the gradient force which is dominant at this scale. To have a high quality factor it is important to have a high signal-to-noise ratio. This can be achieved by using a cooling mechanism to account for the forces exerted on the particle by residual gas molecules. The cooling which is done by moving the trapping beam with a piezo-stage needs a feedback loop in order to work properly. One way to measure the particles position is by putting the particle into an optical cavity which will cause the scattered light to excite a mode inside of the cavity. Part of the mode will be coupled out of the cavity and can be used to determine the position of the particle via the measured intensity and then use this for feedback-cooling. To improve this method this semester thesis had the goal to to find a feasible method for fabricating a microcavity. The semester thesis succeeded in the detailed description of a fabrication process for microcavity mirrors. Stamps and microcavity mirrors were produced. However, while the stamp quality was in agreement with the specified requirements the mirrors themselves had a roughness which was too high ($\si{RMS\approx 1.7\nm}$ instead of $\si{RMS}<0.6\nm$). Efforts to improve the mirror quality were taken but it was not possible to see all of them through during the course of this work.
