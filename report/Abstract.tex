\section*{Abstract}
For exploring the boundaries between the micro- and the macroscopic world we work with mesoscopic glass particles. A nano-scale glass particle can be detached from most environmental influences by putting it into a vacuum and using a laser beam to levitate it. The suspension of the particle does not rely on radiation pressure but on the gradient force which is dominant at this scale. To have a high quality factor it is important to have a high signal-to-noise ratio. This can be achieved by using a cooling mechanism to account for the forces exerted on the particle by residual gas molecules. The cooling which is done by moving the trapping beam with a piezo-stage needs a feedback loop in order to work properly with an accurate measurement of the particles current location. One way to measure the particles position is by adding an optical cavity around the particle which will cause the scattered light to form a mode inside of it. Part of the mode will be coupled out of the cavity and can be used to determine the position of the particle via the measured intensity. To improve this method this semester thesis has the goal of finding a feasible method for fabricating a microcavity. Microcavities have a much smaller mode volume which leads to the particles presence having a much bigger impact on the measured intensity at the detector outside of the cavity.
