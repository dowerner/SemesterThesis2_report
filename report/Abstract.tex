\section*{Abstract}
To explore the boundaries between the micro- and the macroscopic world we work with silica on the nano-scale. A nano-scale glass particle can be detached from all environmental influences by putting it into a vacuum and using a laser beam to levitate it. This resonator has a very high Q-factor and therefore a high Signal-to-Noise ratio which makes it a highly sensitive force sensor, capable of measuring forces as small as $2\cdot 10^{-20}\,\si{N/\sqrt{Hz}}$ \cite{gieseler2013thermal}. The levitation of the particle does not rely on radiation pressure but on the gradient force which is dominant at this scale. In order to have a high quality factor the levitated particle is put in a vacuum to reduce the environmental pressure. In this configuration the particle only subject to residual air molecules causing positional fluctuations. To account for those fluctuations the laser beam is modulated by using a feedback mechanism, which checks the particles position. One way to measure the particles position is by putting the particle into an optical cavity which will cause the scattered light to excite a mode inside of the cavity. The light leaks out of the cavity and can be used to determine the position of the particle. When using cavities for the particle detection we can exploit the Purcell effect which enhances light coupled into the cavity. Furthermore, the light does not need to be collected before or after the trap. To improve this method this semester thesis had the goal to to find a feasible method for fabricating a microcavity. During the semester thesis we managed to develop a promising protocol for microcavity mirror fabrication. Stamps and microcavity mirrors were produced. However, while the stamp quality was in agreement with the specified roughness of $\si{RMS}<0.6\nm$ the mirrors themselves had a roughness which was too high ($\si{RMS\approx 1.7\nm}$). 
