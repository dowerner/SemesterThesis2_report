\chapter{Fabrication}

\section{Requirements}
The main aspect of this project was concerned with finding a process to successfully fabricate a microcavity of medium-high finesse. For this purpose it is necessary to investigate what requirements are given for this cavity to be useful for the trapping experiment it will eventually be used in.
\subsection{Cavity Length}
As discussed in \autoref{ChapSensingFactor}, the sensing factor is a quantity which determines how strong the presence of a glass particle influences the optical properties of the cavity. \autoref{EqSensingFactor} is inversely proportional to the length of the cavity which was explained through the fact that a smaller mode volume will cause the volume of the nano-scaled, dielectric object inside of the cavity to become larger in comparison to the mode volume.\\
Another consideration that has to be made is that the cavity dimensions cannot be made arbitrarily small. The limiting factor in this consideration is not the cavity itself but rather the trapping beam which hits the particle inside the cavity perpendicular to the cavity axis as discussed in \autoref{ChapLaserTrapping}. Making the cavity too small would result in the cavity mirrors clipping the trapping beam which would dramatically reduce its efficiency and cause scattering of energy into undesired modes.
\begin{figure}[H]\label{FigBeamWidth}
	\includesvg[scale=0.5]{source/trapping_beam_width}
	\caption{This figure shows the width of the trapping beam $w_{\si{T}}$ which has to be taken into account when specifying the length of the cavity.}
\end{figure}
As discussed in \autoref{ChapLaserTrapping} the trapping beam can be approximated as a gaussian beam. This makes it easy to estimate what the value of $w_{\si{T}}$ will be. We use the definition of the beam waist of a gaussian beam
\begin{equation}
	w(z)=z\left[1+\left(\frac{z_{\si{R}}}{z}\right)^2\right]
\end{equation}
where $z_{\si{R}}$ is the Rayleigh length. It is also important to note that $w_{\si{T}}=2\cdot w(z)|_{z=z_{\si{T}}}$ since $w(z)$ measures the beams radius and not its diameter. We can also determine $z_{\si{R}}$ since it known that the beam width in the focus will be $d_0=2\cdot w_0 \approx 0.5 \si{\mu m}$.
\begin{equation}
	z_{\si{R}}=\frac{\pi w_0^2}{\lambda}
\end{equation}
If the beam width in relation to the distance from the focus of the beam $z$ is plotted we see the following picture.
\begin{figure}[H]
	\includesvg[scale=1]{source/trapping_beam_diameter}
	\caption{The Figure shows the beam diameter $d(z)=2\cdot w(z)$ in relation to the distance of the beams focus.}
\end{figure}
