\chapter{Introduction}
For the purpose of sensing very tiny masses \cite{chaste2012nanomechanical, yang2006zeptogram}, charges \cite{cleland1998nanometre}, magnetic fields \cite{rugar2004single} or weak forces \cite{stipe2001noncontact, moser2013ultrasensitive}, recent developments in optomechanics \cite{arndt2014testing, aspelmeyer2012quantum, aspelmeyer2014cavity} have brought forth resonators with very high Q-factors which are potentially capable of doing such measurements. Limitations of such resonators are that they are susceptible to temperature fluctuations, dissipation losses as well as thermomechanical noise \cite{ekinci2004ultimate, aspelmeyer2012quantum, postma2005dynamic}. To omit those kinds of problems a different kind of resonator can be used: A levitated nanoparticle in high vacuum \cite{gieseler2013thermal}. Such a particle can achieve a very high Q-factor that is only limited by the collision with residual air molecules. In order for the levitated nanoparticle to act as a resonator with high Q-factor the influence of thermal noise has to be mitigated by vacuum. For the particle to remain levitated and stay in the linear regime, feedback-cooling is used. From the effects that the perturbation of a nanoparticle in a cavity causes, it can be seen that this is one possible way of measuring the required parameters for feedback-cooling \cite{tanjiinteraction, chang2010cavity}. The cavity leaks out the light of the excited mode, generated by scattered light of the particle, hereby allowing to very precisely determine the particles position.\\
The goal of this work is to further improve the feedback cooling mechanism by developing a feasible method to fabricate microcavities. A cavity of small mode volume is essential for detection of the particle. This is due to the fact that the detuning of the cavity resonance is proportional to the ratio of the particle volume and the mode volume \cite{chang2010cavity}. Ultimately, the goal is to use the knowledge of past setups that used larger cavities and build a new setup that is more sensitive to nanoparticles.







