\chapter{Introduction}
For the purpose of sensing very tiny masses, charges, magnetic fields or weak forces, recent developments in optomechanics have brought forth resonators with very high Q-factors which are potentially capable doing such measurements. Limitation of such resonators are that they are susceptible to temperature fluctuations, dissipation losses as well as thermomechanical noise. To omit those kinds of problems a different kind of resonator can be used, a levitated nanoparticle in high vacuum [nphys2798.pdf]. Such a particle can achieve a very high Q-factor that is only limited by the collision with residual air molecules. In order for the levitated nanoparticle to act as a resonator with high Q-factor the influence of thermal noise has to be mitigated by using feedback cooling. It has been shown that a very promising way of measuring the required parameters for said feedback cooling is the placement of the nanoparticle in an optical cavity which couples out the light of the cavity mode, generated by scattered light, hereby allowing to very precisely determine the influence of the nanoparticle on the light.\\
The goal of this work is to further improve the feedback cooling mechanism by developing a feasible method to fabricate microcavities for the nanoparticle to be put inside of. The advantage of very small cavities compared to larger ones is that the difference of the mode volume inside of the cavity is smaller than before which gives the presence of the the very small nanoparticle more significance in terms of the impact of the cavity linewitdth compared to a larger cavity. With this new cavity fabrication method we hope to optimize the feedback cooling mechanism by being able to determine the exact state of the particle inside of the cavity more accurately. Ultimately the goal is to use the knowledge of past setups that used larger cavities and build a new setup that allows the full exploration of the possibilities that this new configuration enables us to do.







